\chapter*{РЕФЕРАТ}
\addcontentsline{toc}{chapter}{РЕФЕРАТ}

Научно-исследовательская работа 29с., 1 табл., 8 ист., 1 прил.

КОМПЬЮТЕРНАЯ ЛИНГВИСТИКА, МАШИННЫЙ ПЕРЕВОД, МЕТОДЫ МАШИННОГО ПЕРЕВОДА, СТАТИСТИЧЕСКИЙ МАШИННЫЙ ПЕРЕВОД, МАШИННЫЙ ПЕРЕВОД НА ОСНОВЕ ПРАВИЛ

\textbf{Цель} --- провести обзор методов машинного перевода.

В процессе работы проводилась классификация существующих методов машинного перевода, а также сравнительный анализ данных методов по сформулированным критериям оценки.

\textbf{Результаты} --- статистический метод машинного перевода имеет большее количество достоинств, чем метод на основе правил, однако для некоторых задач его недостатки являются критичными, по этой причине оба метода до сих пор активно используются.


\chapter*{ВВЕДЕНИЕ}
\addcontentsline{toc}{chapter}{ВВЕДЕНИЕ}

На сегодняшний день существует явная тенденция развития международных коммуникаций как в рабочих сферах, так и за их пределами. Для качественного взаимодействия между людьми, компаниями крайне необходима возможность выстраивать коммуникацию с наибольшей точностью обработки передаваемой или получаемой информации. Данную возможность могут представлять профессиональные переводчики, однако этот способ не является доступным для большинства, по этой причине актуальны методы машинного перевода и, в частности, важной является проблема качества машинного перевода.

\textbf{Цель данной научно-исследовательской работы} --- провести обзор методов машинного перевода.

Для достижения поставленной цели необходимо решить следующие задачи:
\begin{itemize}[label=---]
	\item описать существующие методы машинного перевода;
    \item классифицировать рассмотренные методы;
    \item сформулировать критерии сравнения методов машинного перевода;
	\item сравнить методы на основании выделенных критериев;
	\item описать результаты сравнения рассмотренных алгоритмов.
\end{itemize}